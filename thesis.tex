%%%%%%%%%%%%%%%%%%%%%%%%%%%%%%%%%%%%%%%%%%%%%%%%%%%%%%%%%%%%%%%%%%%%
%%%%%%%%%%%%%%%%%%%%%%%%%%%%%%%%%%%%%%%%%%%%%%%%%%%%%%%%%%%%%%%%%%%%
%%                                                                %%
%% An example for writting your thesis using LaTeX                %%
%% Original version by Luis Costa,  changes by Perttu Puska       %%
%% Support for Swedish added 15092014                             %%
%%                                                                %%
%% This example consists of the files                             %%
%%         thesistemplate.tex (versio 2.01)                       %%
%%         opinnaytepohja.tex (versio 2.01) (for text in Finnish) %%
%%         aaltothesis.cls (versio 2.01)                          %%
%%         kuva1.eps                                              %%
%%         kuva2.eps                                              %%
%%         kuva1.pdf                                              %%
%%         kuva2.pdf                                              %%
%%                                                                %%
%%                                                                %%
%% Typeset either with                                            %%
%% latex:                                                         %%
%%             $ latex opinnaytepohja                             %%
%%             $ latex opinnaytepohja                             %%
%%                                                                %%
%%   Result is the file opinnayte.dvi, which                      %%
%%   is converted to ps format as follows:                        %%
%%                                                                %%
%%             $ dvips opinnaytepohja -o                          %%
%%                                                                %%
%%   and then to pdf as follows:                                  %%
%%                                                                %%
%%             $ ps2pdf opinnaytepohja.ps                         %%
%%                                                                %%
%% Or                                                             %%
%% pdflatex:                                                      %%
%%             $ pdflatex opinnaytepohja                          %%
%%             $ pdflatex opinnaytepohja                          %%
%%                                                                %%
%%   Result is the file opinnaytepohja.pdf                        %%
%%                                                                %%
%% Explanatory comments in this example begin with                %%
%% the characters %%, and changes that the user can make          %%
%% with the character %                                           %%
%%                                                                %%
%%%%%%%%%%%%%%%%%%%%%%%%%%%%%%%%%%%%%%%%%%%%%%%%%%%%%%%%%%%%%%%%%%%%
%%%%%%%%%%%%%%%%%%%%%%%%%%%%%%%%%%%%%%%%%%%%%%%%%%%%%%%%%%%%%%%%%%%%

%% Uncomment one of these:
%% the 1st when using pdflatex, which directly typesets your document in pdf (use jpg or pdf figures), or the 2nd when producing a ps file (use eps figures, don't use ps figures!).
\documentclass[english,12pt,a4paper,pdftex,sci,utf8]{aaltothesis}
%\documentclass[english,12pt,a4paper,dvips]{aaltothesis}

%% To the \documentclass above:
%% specify your school: arts, biz, chem, elec, eng, sci
%% specify the character encoding scheme used by your editor: utf8, latin1

%% Use one of these if you write in Finnish (see the Finnish template):
%\documentclass[finnish,12pt,a4paper,pdftex,elec,utf8]{aaltothesis}
%\documentclass[finnish,12pt,a4paper,dvips]{aaltothesis}

\usepackage{graphicx}

%% Use this if you write hard core mathematics, these are usually needed
\usepackage{amsfonts,amssymb,amsbsy}

%% Use the macros in this package to change how the hyperref package below typesets its hypertext -- hyperlink colour, font, etc. See the package documentation. It also defines the \url macro, so use the package when not using the hyperref package.
%\usepackage{url}

%% Use this if you want to get links and nice output. Works well with pdflatex.
\usepackage{hyperref}
\hypersetup{pdfpagemode=UseNone, pdfstartview=FitH,
  colorlinks=true,urlcolor=red,linkcolor=blue,citecolor=black,
  pdftitle={Default Title, Modify},pdfauthor={Your Name},
  pdfkeywords={Modify keywords}}


%% All that is printed on paper starts here
\begin{document}

%% Change the school field to specify your school if the automatically set name is wrong
% \university{aalto-yliopisto}
% \university{aalto University}
% \school{Sähkötekniikan korkeakoulu}
% \school{School of Electrical Engineering}

%% Only for B.Sc. thesis: Choose your degree programme.
%\degreeprogram{Electronics and electrical engineering}

%% ONLY FOR M.Sc. AND LICENTIATE THESIS: Specify your department,
%% professorship and professorship code.
\department{Department of Computer Science}
\professorship{Computer Science}
%%

%% Valitse yksi näistä kolmesta
%%
%% Choose one of these:
%\univdegree{BSc}
\univdegree{MSc}
%\univdegree{Lic}

%% Your own name (should be self explanatory...)
\author{Aarni Halinen}

%% Your thesis title comes here and again before a possible abstract in
%% Finnish or Swedish . If the title is very long and latex does an
%% unsatisfactory job of breaking the lines, you will have to force a
%% linebreak with the \\ control character.
%% Do not hyphenate titles.
%%
\thesistitle{Kubernetes inter-pod container isolation}

\place{Espoo}

%% For B.Sc. thesis use the date when you present your thesis.
\date{1.6.2023}

%% B.Sc. or M.Sc. thesis supervisor
%% Note the "\" after the comma. This forces the following space to be a normal interword space, not the space that starts a new sentence.
%% This is done because the fullstop isn't the end of the sentence that should be followed by a slightly longer space but is to be followed by a regular space.
\supervisor{Prof.\ Mario Di Francesco} %{Prof.\ Pirjo Professori}

%% B.Sc. or M.Sc. thesis advisors(s). You can give upto two advisors in this template. Check with your supervisor how many official advisors you can have.
\advisor{M.Sc.\ (Tech.)\ José\ Luis\ Martin\ Navarro}
\advisor{M.Sc.\ (Tech.)\ Jacopo\ Bufalino}

%% Aalto logo: syntax:
%% \uselogo{aaltoRed|aaltoBlue|aaltoYellow|aaltoGray|aaltoGrayScale}{?|!|''}
%%
%% Logo language is set to be the same as the document language.
%% Logon kieli on sama kuin dokumentin kieli
\uselogo{aaltoRed}{''}

%% Create the coverpage
\makecoverpage

%% Note that when writting your master's thesis in English, place
%% the English abstract first followed by the possible Finnish abstract

%% English abstract.
%% All the information required in the abstract (your name, thesis title, etc.)
%% is used as specified above.
%% Specify keywords
\keywords{Kubernetes, Container, Docker, Security}
%% Abstract text
\begin{abstractpage}[english]
  Your abstract in English. Try to keep the abstract short; approximately
  100 words should be enough. The abstract explains your research topic,
  the methods you have used, and the results you obtained.
  Your abstract in English. Try to keep the abstract short; approximately
  100 words should be enough. The abstract explains your research topic,
  the methods you have used, and the results you obtained.

  Your abstract in English. Try to keep the abstract short; approximately
  100 words should be enough. The abstract explains your research topic,
  the methods you have used, and the results you obtained.
  Your abstract in English. Try to keep the abstract short; approximately
  100 words should be enough. The abstract explains your research topic,
  the methods you have used, and the results you obtained.
\end{abstractpage}

%% Force a new page so that the possible English abstract starts on a new page
\newpage
%
%% Abstract in Finnish.  Delete if you don't need it.
% \thesistitle{Opinnäyteohje}
% \advisor{TkT Olli Ohjaaja}
% \degreeprogram{Electronics and electrical engineering}
% \department{Radiotieteen ja -tekniikan laitos}
% \professorship{Piiriteoria}
% %% Avainsanat
% \keywords{Vastus, Resistanssi,\\ Lämpötila}
% %% Tiivistelmän tekstiosa
% \begin{abstractpage}[finnish]
%   Tiivistelmässä on lyhyt selvitys (noin 100 sanaa)
%   kirjoituksen tärkeimmästä sisällöstä: mitä ja miten on tutkittu,
%   sekä mitä tuloksia on saatu.
%   Tiivistelmässä on lyhyt selvitys (noin 100 sanaa)
%   kirjoituksen tärkeimmästä sisällöstä: mitä ja miten on tutkittu,
%   sekä mitä tuloksia on saatu.

%   Tiivistelmässä on lyhyt selvitys (noin 100 sanaa)
%   kirjoituksen tärkeimmästä sisällöstä: mitä ja miten on tutkittu,
%   sekä mitä tuloksia on saatu.
%   Tiivistelmässä on lyhyt selvitys (noin 100 sanaa)
%   kirjoituksen tärkeimmästä sisällöstä: mitä ja miten on tutkittu,
%   sekä mitä tuloksia on saatu.
%   Tiivistelmässä on lyhyt selvitys (noin 100 sanaa)
%   kirjoituksen tärkeimmästä sisällöstä: mitä ja miten on tutkittu,
%   sekä mitä tuloksia on saatu.
% \end{abstractpage}

% \newpage

\mysection{Preface}
I want to thank Professor Pirjo Professori
and my instructor Olli Ohjaaja for their
good and poor guidance.\\

\vspace{5cm}
Otaniemi, 16.1.2015

\vspace{5mm}
{\hfill Eddie E.\ A.\ Engineer \hspace{1cm}}

%% Force new page after preface
\newpage

%% Table of contents.
\thesistableofcontents

%% Symbols and abbreviations
\mysection{Symbols and abbreviations}

\subsection*{Symbols}

\begin{tabular}{ll}
$\uparrow$       & electron spin direction up\\
$\downarrow$     & electron spin direction down
\end{tabular}

\subsection*{Operators}

\begin{tabular}{ll}
$\nabla \times \mathbf{A}$              & curl of vectorin $\mathbf{A}$\\
\end{tabular}

\subsection*{Abbreviations}

\begin{tabular}{ll}
K8s         & Kubernetes \\
STRIDE      & an object-oriented analog circuit simulator and design tool \\
\end{tabular}


%% Tweaks the page numbering to meet the requirement of the thesis format:
%% Begin the pagenumbering in Arabian numerals (and leave the first page
%% of the text body empty, see \thispagestyle{empty} below).
%% Additionally, force the actual text to begin on a new page with the
%% \clearpage command.
%% \clearpage is similar to \newpage, but it also flushes the floats (figures
%% and tables).
%% There is no need to change these
\cleardoublepage
\storeinipagenumber
\pagenumbering{arabic}
\setcounter{page}{1}

\section{Introduction}

%% Leave first page empty
\thispagestyle{empty}

T\"am\"an tekstin l\"ahteen\"a oleva tiedosto on opinn\"aytteen
pohja, jota voi k\"aytt\"a\"a kandidaatinty\"oss\"a, diplomity\"oss\"a ja
lisensiaatinty\"oss\"a. Tekstin
l\"ahteen\"a oleva tiedosto on kirjoitettu  \LaTeX-tiedoston rakenteen
opiskelemista ajatellen. Tiedoston kommentit sis\"alt\"av\"at
tietoa, joka on hy\"odyllist\"a opinn\"aytett\"a kirjoitettaessa.

%% Esimerkki luettelosta. Lyhyt ajatusviiva on k\"ayt\"oss\"a
%% luettelossa, ja se on pituudeltaan
%% en dash. Merkit\"a\"an latex-koodissa --.
Johdanto selvitt\"a\"a samat asiat kuin tiivistelm\"a, mutta
laveammin. Johdannossa kerrotaan yleens\"a seuraavat asiat

\begin{itemize}
\item[--]Tutkimuksen taustaa ja tutkimusaiheen yleisluonteinen esittely
\item[--]Tutkimuksen tavoitteet
\item[--]P\"a\"akysymys ja osaongelmat
\item[--]Tutkimuksen rajaus ja keskeiset k\"asitteet.
\end{itemize}

Lyhyiden opinn\"aytteiden johdannot ovat yleens\"a liian pitki\"a, joten
johdannon paisuttamista on v\"altett\"av\"a. Diplomity\"oh\"on sopii johdanto,
joka on 2--4 sivua. %% t\"ass\"a on my\"os lyhyt ajatusviiva l. en dash.
Kandidaatinty\"on johdannon on oltava diplomity\"on
johdantoa lyhyempi. Sopivasti tiivistetty johdanto ei kaipaa alaotsikoita.


%% In a thesis, every section starts a new page, hence \clearpage
\clearpage

\section{Background}

% T\"ass\"a osassa selvitet\"a\"an, mit\"a tutkimuksen kohteena olevasta
% aiheesta tiedet\"a\"an entuudestaan. Selvityksen tulee kattaa
% tasapainoisesti koko tutkimuskentt\"a.

% Kun opinn\"aytety\"ot\"a kirjoitetaan, on noudatettava
% ohjeita, jotka koskevat opinn\"aytteen rakennetta,
% k\"ayt\"ant\"oj\"a, muotoseikkoja sek\"a ulkoasua. Esitell\"a\"an n\"ait\"a
% ohjeita tarkemmin.

% \subsection*{Rakenne}

% Opinn\"aytteen rakenteen tulee olla hyv\"an tieteellisen
% kirjoittamisen k\"ayt\"ann\"on mukainen ja sis\"alt\"a\"a v\"ahint\"a\"an seuraavat
% osat:

% \begin{enumerate}
% \item Nimi\"olehti
% \item Tiivistelm\"a
% \item Sis\"allysluettelo
% \item Symboli- ja lyhenneluettelo
% \item \label{a} Johdanto
% \item  Aikaisempi tutkimus. Ty\"on luonteen niin vaatiessa otsikko voi olla my\"os
%         >>Teoreettinen tausta>>  tai n\"aiden otsikoiden yhdistelm\"a.
% \item Tutkimusaineisto ja -menetelm\"at %% yhdysmerkki - eli tavuviiva.
% \item Tulokset
% \item \label{o} Tarkastelu. Ty\"on luonteen niin vaatiessa otsikko voi
%       olla my\"os >>Johtop\"a\"at\"okset>> tai >>Yhteenveto>>
%       tai edell\"a mainittujen otsikoiden yhdistelm\"a.
% \item L\"ahteet
% \item Liitteet.
% \end{enumerate}

\subsection*{Microservices?}
\clearpage

\subsection*{Containerization and Docker}
\subsubsection*{Breakout (Priviledge escalation)}
\begin{enumerate}
  \item Mounted docker socket
  \item Priviledged container
\end{enumerate}

\subsection*{Kubernetes}
\subsubsection*{apiserver}
\subsubsection*{etcd}
\subsubsection*{scheduler}
\subsubsection*{controller-manager}

\subsection*{Kubernetes security boundaries}
\subsubsection*{Namespaces}
\subsubsection*{Pods}

\subsection*{Security analysis?}
\subsubsection*{STRIDE}

\subsection*{Sivut ja kirjaintyypit}

Opinn\"aytteen tulee olla kirjoitettu koneella tai
tekstink\"asittelyohjelmalla yksipuolisesti A4-kokoiselle paperille.
Kandidaatinty\"on tekstiosan sopiva pituus on noin 15--20 sivua ja
diplomity\"on noin 60 sivua. Ty\"ot\"a ei ole syyt\"a tarpeettomasti pident\"a\"a.

Opinn\"aytteen tekstiosan kirjaintyypin tulee olla antiikva eli
%% esimerkki pakkotavutuksesta; "serif-tyyppinen" on tavutuksen kannalta
%% hankala, joten pakkotavutetaan se.
serif\--tyyp\-pi\-nen ja lis\"aksi kursivoimaton, lihavoimaton sek\"a kooltaan 12
pistett\"a (kuten t\"ass\"a esityksess\"a). Groteskeja eli \textsf{Sans
  serif}-tyyppisi\"a kirjaintyyppej\"a (kuten Helvetica tai Arial) ei saa
k\"aytt\"a\"a varsinaisessa tekstiss\"a, mutta otsikoissa n\"ait\"a voidaan
k\"aytt\"a\"a.  Otsikoissa voidaan k\"aytt\"a\"a kooltaan edell\"a mainittua
suurempaa kirjaintyyppi\"a sek\"a tyylikeinoja, kuten lihavointia tai
kursivointia.  Tekstiss\"a samantasoisten otsikoiden on kuitenkin oltava
tyylilt\"a\"an ja kirjainlajeiltaan yhtenev\"aisi\"a.
%% Esimerkki taulukosta
\begin{table}[htb]
%% Taulukon teksti
\caption{Taulukoissa ja kuvissa kirjaintyypin voi valita
tarkoituksenmukaisesti, mutta kuva- ja taulukkoteksteiss\"a tulee
k\"aytt\"a\"a samaa kirjaintyyppi\"a kuin varsinaisessa tekstiss\"a.
Huomaa taulukon numeroinnin sijoittuminen taulukon yl\"apuolelle. \label{taulukko1}}
\begin{center}
\fbox{
\begin{tabular}{c|l|r}
\textbf{A} & 1 & $e^{j \omega t}$ \\ \hline
\textsf{B} & 2 & ${\mathfrak R}(c)$ \\ \hline
\texttt{C} & 3 & $ a \in \mathbb{A}$
\end{tabular}
}
\end{center}
\end{table}

Opinn\"aytteen vasen marginaali (sidonnan puoli) on
35~mm % t\"ass\"a ~ muodostaa ns. yhdist\"av\"an v\"alily\"onnin
ja oikea 25~mm. Yl\"amarginaali on 25~mm. Leip\"atekstin korkeus on
enimmill\"a\"an 230mm. T\"am\"an opinn\"aytepohjan marginaalien pit\"aisi olla
paperille tulostettuna oikein, mutta tulostimesta ja paperista
riippuen voi esiinty\"a yhden tai kahden millimetrin suuruisia eroja.
%% Jos k\"a\"ann\"at t\"am\"an tekstin pdflatex-komennolla ja tulostat sen katselu-
%% ohjelmasta, toteat todenn\"ak\"oisesti em. mittojen poikkeavan enemm\"an
%% kuin 1-2 mm.
%% T\"am\"a on seurausta pdf-tiedoston erilaisesta kirjaintyyppim\"a\"arityksest\"a.
%% Korkeatasoista painoty\"ot\"a varten k\"ayt\"a vain latex-komentoa ja
%% tulosta postscript-muotoon k\"a\"annetyst\"a tiedostosta.
\subsection*{Asemointi}

%% Muutos vanhaan ohjeeseen verrattuna: aikaisemmassa ohjeessa
%% kehotettiin k\"aytt\"am\"a\"an vasensuora-asettelua, mutta t\"ass\"a
%% ohjeessa ollaan luovuttu tuosta vaatimuksesta ja siirrytty
%% huoliteltumpaan, painotuotteenomaisempaan suuntaan.
Tekstiosan tekstiss\"a k\"aytet\"a\"an kappaleiden erottamiseen sisennyst\"a,
mutta ensimm\"aist\"a otsikon, v\"aliotsikon tai muun katkon j\"alkeist\"a
kappaletta ei sisennet\"a. Jos kuva tai muu katko tulee kappaleiden
v\"aliin, suositellaan katkon j\"alkeisen kappaleen sisent\"amist\"a.

Mik\"ali oikea reuna halutaan tasata, tulee k\"aytt\"a\"a tavutusta ja lis\"aksi
tarkistaa, ettei tekstiin j\"a\"a lukemista h\"airitsevi\"a pitki\"a sanav\"alej\"a. Jos
k\"ayt\"at opinn\"aytteen tekemisess\"a \LaTeX-j\"arjestelm\"a\"a,
t\"am\"a asia hoituu automaattisest.

Opinn\"aytteen riviv\"ali on 1, mik\"a on my\"os t\"am\"an opinn\"aytepohjan k\"ayt\"ant\"o.
Kappaleiden tulee yleens\"a olla ainakin kolmen rivin pituisia, mutta
my\"os liian pitki\"a kappaleita tulee v\"altt\"a\"a.  T\"ass\"a opinn\"aytepohjassa
ei tekstin luonteen vuoksi voida t\"aysin toteuttaa kappaleen pituutta koskevia
vaatimuksia.

Yksitt\"aisi\"a, kappaleen p\"a\"att\"avi\"a tai aloittavia rivej\"a sivun alussa
tai lopussa on v\"altett\"av\"a koko ty\"oss\"a, my\"os luetteloissa ja
liitteiss\"a.

\subsection*{Numerointi}

Opinn\"aytteen jokainen osa alkaa uudelta sivulta. Alaosa aloittaa uuden
sivun vain edellisen sivun t\"aytytty\"a.

Ty\"on osat numeroidaan siten, ett\"a johdanto on ensimm\"ainen numeroitava
osa. Osien numeroinnissa k\"aytet\"a\"an arabialaisia numeroita.

Nimi\"olehti, tiivistelm\"a, esipuhe, sis\"allysluettelo ja symboli- ja
lyhenneluettelo numeroidaan esipuheesta tai t\"am\"an puuttuessa
ensimm\"aiselt\"a luettelosivulta alkaen roomalaisin numeroin.

Sivunumerointi alkaa toiselta varsinaiselta tekstisivulta, ja
sivunumeroinnissa k\"aytet\"a\"an arabialaisia numeroita.

L\"ahdeluettelo alkaa uudelta sivulta. L\"ahdeluettelon sivunumerointi
jatkuu viimeisest\"a tekstisivusta.

Jokainen liite alkaa uudelta sivulta. Liitteiden sivunumerointi
jatkuu viimeisest\"a l\"ahdeluettelon sivusta.

Sivunumero sijoitetaan sivun yl\"areunaan.

Matemaattiset kaavat numeroidaan arabialaisin
numeroin. Kaavanumerointi ei saa katketa osien v\"aliss\"a (eik\"a niin
tapahdukaan, jos k\"ayt\"at t\"at\"a opinn\"aytepohjaa). Kaikkia kaavoja ei tarvitse
numeroida, vaan kirjoittaja voi k\"aytt\"a\"a harkintaa numeroinnin
tarpeellisuudessa.  Liitteiss\"a olevat kaavat numeroidaan siten, ett\"a
liitteen ajatellaan muodostavan numeroinnin kannalta itsen\"aisen ja
yhten\"aisen kokonaisuuden. Kaavan numero sijoitetaan oikealle puolelle
alla olevan esimerkin mukaisesti
\begin{equation}
D(xy) = (Dx)y + x(Dy),  \hspace{3em} x,y \in \mathbb{A}.
\end{equation}
%% Kaavojen j\"alkeen ei yleens\"a laiteta sisennyst\"a.
Kaikki kuvat ja taulukot numeroidaan erillisen juoksevan numeroinnin
mukaisesti kuten taulukosta \ref{taulukko1} ja kuvasta \ref{kuva1} k\"ay
ilmi.  Liitteiss\"a olevat kuvat ja taulukot numeroidaan siten, ett\"a
liitteen ajatellaan muodostavan numeroinnin kannalta itsen\"aisen ja
yhten\"aisen kokonaisuuden. Liitteiss\"a \ref{LiiteA} ja \ref{LiiteB} on
esimerkkej\"a kaavojen (kaavat \ref{liitekaava1}--\ref{liitekaava2} tai
kaavat \ref{liitekaava3}--\ref{liitekaava4}), kuvien (kuva
\ref{liitekuva}) ja taulukoiden (taulukko \ref{liitetaulukko})
numeroimisesta.  Liitteet numeroidaan suuraakkosin (esimerkiksi Liite
A, Liite B tai pelk\"ast\"a\"an A, B).
%% T\"ass\"a esimerkki kuva1.pdf -nimisen tiedoston tuomisesta kuvaksi.
%% Komento \inclugraphics[parametrit]{argumentti} tuo kuvan.
%% Komento \centering pakottaa kuvan keskelle.
%% Komento \caption luo kuvatekstin ja sen numeroinnin
%% Parametrit htb pakottavat kuvan suunnilleen siihen
%% kohtaan, miss\"a se esiintyy tekstin l\"ahdekoodissa
\begin{figure}[htb]
\centering \includegraphics[height=5cm]{kuva1}
\caption{T\"am\"a on esimerkki numeroidusta kuvatekstist\"a. \label{kuva1}}
\end{figure}

\subsection*{L\"ahdeviittausten k\"aytt\"o}

L\"ahdeviittaukset tulee tehd\"a huolellisesti ja johdonmukaisesti
numeroviitej\"arjestelm\"an mukaisesti. Numeroviitteet j\"arjestet\"a\"an
l\"ahdeluetteloon viittausj\"arjestykseen, mutta jos l\"ahdeluettelo
on hyvin laaja (useita sivuja), j\"arjestet\"a\"an viitteet p\"a\"asanan
mukaiseen aakkosj\"arjestykseen. Alaviitej\"arjestelm\"a\"a
\footnote{My\"osk\"a\"an alaviitteen\"a olevia kommentteja \underline{ei} suositella
k\"aytett\"aviksi.} ei k\"aytet\"a.

Viitteen sijoittelussa noudatetaan seuraavia s\"a\"ant\"oj\"a:
Jos viite kohdistuu vain yhteen virkkeeseen tai virkkeen
osaan, viite \cite{Kauranen} sijoitetaan virkkeen sis\"a\"an ennen virkett\"a
p\"a\"att\"av\"a\"a pistett\"a. Jos taas viite koskee tekstin useampaa
virkett\"a tai kokonaista kappaletta, sijoitetaan viite kappaleen loppuun
pisteen j\"alkeen. \cite{Kauranen}

\subsection*{L\"ahdeluettelo}

L\"ahdeluettelossa esiintyy tavallisesti seuraavassa esitett\"avi\"a
l\"ahteit\"a, joista on numeroviitej\"arjestelm\"ass\"a ilmoitettava
asianomaisessa kohdassa vaaditut tiedot.

%% Esimerkki korostamisesta. Lihavoinnin sijasta on tyylikk\"a\"amp\"a\"a
%% ja luettavampaa k\"aytt\"a\"a kursiivia.
\textit{Kirjasta} ilmoitetaan seuraavat tiedot:

\begin{itemize}
\item[--]tekij\"at
\item[--]julkaisun nimi
\item[--]painos, jos useita
\item[--]kustannuspaikka
\item[--]julkaisija tai kustantaja
\item[--]julkaisuaika
\item[--]mahdollinen sarjamerkint\"o.
\end{itemize}

Viitteet \cite{Kauranen}--\cite{Koblitz} ovat esimerkkej\"a kirjan
esitt\"amisest\"a l\"ahdeluettelossa. Viite \cite[s.\ 83--124]{Koblitz} on
esimerkki l\"ahdeluettelossa esiintyv\"an kirjan tiettyjen sivujen
esitt\"amisest\"a tekstiss\"a.

\textit{Artikkelista} kausijulkaisussa ilmoitetaan seuraavat tiedot:

\begin{itemize}

\item[--]tekij\"at
\item[--]artikkelin nimi
\item[--]kausijulkaisun nimi
\item[--]julkaisuvuosi
\item[--]kausijulkaisun volyymi tai ilmestymisvuosi
\item[--]kausijulkaisun numero
\item[--]sivut, joilla artikkeli on.
\end{itemize}

Viitteet \cite{bcs}--\cite{Deschamps} ovat esimerkkej\"a artikkelin
esitt\"amisest\"a l\"ahdeluettelossa.

\textit{Kokoomateoksen luvusta tai osasta} ilmoitetaan seuraavat tiedot:

\begin{itemize}
\item[--]luvun tai osan tekij\"at
\item[--]luvun tai osan nimi
\item[--]maininta >>Teoksessa>>
\item[--]koko teoksen toimittajat sek\"a maininta >>(toim.)>>
\item[--]koko teoksen tai konferenssin nimi
\item[--]konferenssiesitelm\"an kyseess\"a ollessa sen pitopaikka ja -aika
\item[--]painos, jos useita
\item[--]kustannuspaikka
\item[--]julkaisija tai kustantaja, jos aihetta t\"am\"an ilmoittamiseen on
\item[--]julkaisuaika
\item[--]sivut, joilla luku tai osa on
\item[--]mahdollinen sarjamerkint\"a.
\end{itemize}

Viitteet \cite{Sihvola}--\cite{Lindblom} ovat esimerkkej\"a
kokoomateoksen luvun tai osan esitt\"amisest\"a l\"ahdeluettelossa.

\textit{Opinn\"aytety\"ost\"a} ilmoitetaan seuraavat tiedot:

\begin{itemize}
\item[--]tekij\"a
\item[--]ty\"on nimi
\item[--]opinn\"aytety\"on tyyppi
\item[--]oppilaitoksen nimi
\item[--]osaston, laitoksen tai ohjelman nimi
\item[--]oppilaitoksen sijaintipaikka
\item[--]vuosiluku.
\end{itemize}

Viitteet \cite{Miinusmaa}--\cite{Lonnqvist} ovat esimerkkej\"a
opinn\"aytteen esitt\"amisest\"a l\"ahdeluettelossa.

\textit{Standardista} ilmoitetaan seuraavat tiedot:

\begin{itemize}
\item[--]standardin tunnus ja numero
\item[--]standardin nimi
\item[--]painos, mik\"ali ei ole ensimm\"ainen
\item[--]julkaisupaikka
\item[--]julkaisija
\item[--]julkaisuvuosi
\item[--]sivum\"a\"ar\"a.
\end{itemize}
Viite \cite{sfs} on esimerkki standardin esitt\"amisest\"a opinn\"aytteen
l\"ahdeluettelossa.

\textit{Haastattelusta} ilmoitetaan seuraavat tiedot:

\begin{itemize}
\item[--]haastatellun henkil\"on nimi
\item[--]haastatellun henkil\"on arvo tai asema
\item[--]haastatellun henkil\"on edustama organisaatio
\item[--]organisaation osoite
\item[--]maininta siit\"a, ett\"a kyseess\"a on haastattelu ja haastattelun
p\"aiv\"am\"a\"ar\"a.
\end{itemize}

Viite \cite{haastattelu} on esimerkki
haastattelun esitt\"amisest\"a l\"ahdeluettelossa.

Osa s\"ahk\"oisess\"a muodossa olevista artikkeleista on saatavissa my\"os
painettuina. \textit{Vain verkosta saatavissa olevasta artikkelista} esitet\"a\"an
seuraavat tiedot:

\begin{itemize}
\item[--]tekij\"at
\item[--]artikkelin nimi
\item[--]kausijulkaisun nimi
\item[--]viestintyyppi
\item[--]laitos tai volyymi
\item[--]kausijulkaisun yksitt\"aist\"a osaa koskeva merkint\"a tai numero
\item[--]julkaisuvuosi tai maininta >>P\"aivitetty>> ja p\"aivitysaika
\item[--]maininta >>Viitattu>> ja viittaamisen ajankohta
\item[--]maininta >>Saatavissa>> ja URL tai
        maininta >>DOI>> ja DOI-numero (DOI=Digital Object Identifier).
\end{itemize}

Viitteet \cite{Ribeiro}--\cite{kone} ovat esimerkkej\"a s\"ahk\"oisess\"a
muodossa olevan artikkelin esitt\"amisest\"a opinn\"aytteen
l\"ahdeluettelossa.  Viitteet \cite{Ribeiro} ja \cite{Stieber} ovat
saatavissa sek\"a painettuna ett\"a verkosta, joten viitteiden esitystapa
mukailee painetun artikkelin viitteen esitystapaa, mutta sen lis\"aksi
kerrotaan julkaisun olevan verkkolehti ja lehden olevan saatavissa
my\"os painettuna.  Viite \cite{kone} on saatavissa vain verkosta ja
siit\"a esitet\"a\"an yll\"a vaaditut tiedot.

Valitettavasti s\"ahk\"oisess\"a muodosssa olevasta artikkelista ei ole aina
saatavissa lai\-tos-, volyymi- tai numerotietoja.

\textit{S\"ahk\"oisess\"a muodossa olevasta opinn\"aytety\"ost\"a} ilmoitetaan
seuraavat tiedot:

\begin{itemize}
\item[--]tekij\"a
\item[--]ty\"on nimi
\item[--]viestintyyppi
\item[--]opinn\"aytety\"on tyyppi
\item[--]oppilaitoksen nimi
\item[--]osaston, laitoksen tai ohjelman nimi
\item[--]oppilaitoksen sijaintipaikka
\item[--]vuosiluku
\item[--]viittamisen ajankohta
\item[--]maininta >>Saatavissa>> ja URL tai
        maininta >>DOI>> ja DOI-numero.
\end{itemize}

Viite \cite{Adida} on esimerkki s\"ahk\"oisess\"a muodossa olevan
opinn\"aytteen esitt\"amisest\"a l\"ahdeluettelossa.

Viite \cite{viittaaminen} on esimerkki itsen\"aisen kirjoituksen sis\"alt\"av\"ast\"a
verkkosivusta. T\"allainen l\"ahde on rinnastettavissa erillisteokseen.
\textit{Verkkosivusta} esitet\"a\"an tiedot:

\begin{itemize}
\item[--] tekij\"at
\item[--] otsikko
\item[--] maininta >>P\"aivitetty>> ja p\"aivitysaika
\item[--] maininta >>Viitattu>> ja viittaamisen ajankohta
\item[--] Maininta >>Saatavissa>> ja URL.
\end{itemize}

Joskus verkkosivun kirjoitus on jaettu useammalle sivulle, jolloin
l\"ahdeluetteloon kirjataan vain sellainen verkko-osoite, joka koskee
koko kirjoitusta tai sen etusivua, ellei sitten
todella tarkoiteta kirjoituksen yksitt\"aist\"a sivua.

\subsection*{Muuta huomioitavaa l\"ahdeluettelossa}

%% Muutos vanhoihin ohjeisiin koskien kielt\"a.
L\"ahdeluettelossa ty\"on ja julkaisun nimi kirjoitetaan alkuper\"aisess\"a
muodossaan. Julkaisijan kotipaikka kirjoitetaan alkukielisess\"a
muodossaan.

Viittamista koskevassa suomalaisessa standardissa
SFS 5342 \cite{sfs} vaaditaan julkaisuista ilmoitettavaksi my\"os ISBN- tai
ISSN-numerot, mutta n\"aiss\"a opinn\"ayteohjeissa ei ISBN- ja
ISSN-numeroita vaadita.

\clearpage

\section{Research material and methods}
%\section{Tutkimusaineisto ja -menetelm\"at}

T\"ass\"a osassa kuvataan k\"aytetty tutkimusaineisto ja
tutkimuksen metodologiset valinnat, sek\"a
kerrotaan tutkimuksen toteutustapa ja k\"aytetyt menetelm\"at.

\clearpage

\section{Results}
%\section{Tulokset}

T\"ass\"a osassa esitet\"a\"an tulokset ja vastataan tutkielman alussa
esitettyihin tutkimuskysymyksiin. Tieteellisen kirjoitelman
arvo mitataan t\"ass\"a osassa esitettyjen tulosten perusteella.

%% Huomaa seuraavassa kappaleessa lainausmerkkien ulkopuolella piste,
%% koska piste ei lopeta lainattua tekstinp\"atk\"a\"a.
%% Jos lainattu tekstinp\"atk\"a loppuu v\"alimerkkiin, tulee v\"alimerkki
%% lainausmerkkien sis\"alle:
%% "Et tu, Brute?" sanoi Caesar kuollessaan.
Tutkimustuloksien merkityst\"a on aina syyt\"a arvioida ja tarkastella
kriittisesti.  Joskus tarkastelu voi olla t\"ass\"a osassa, mutta se
voidaan my\"os j\"att\"a\"a viimeiseen osaan, jolloin viimeisen osan nimeksi
tulee >>Tarkastelu>>. Tutkimustulosten merkityst\"a voi arvioida my\"os
>>Johtop\"a\"at\"okset>>-otsikon alla viimeisess\"a osassa.

T\"ass\"a osassa on syyt\"a my\"os arvioida tutkimustulosten luotettavuutta.
Jos tutkimustulosten merkityst\"a arvioidaan >>Tarkastelu>>-osassa,
voi luotettavuuden arviointi olla my\"os siell\"a.

\clearpage

\section{Summary}
%\section{Yhteenveto}

Opinn\"aytteen tekij\"a vastaa siit\"a, ett\"a opinn\"ayte on t\"ass\"a dokumentissa
ja opinn\"aytteen tekemist\"a k\"asittelevill\"a luennoilla sek\"a
harjoituksissa annettujen ohjeiden mukainen muotoseikoiltaan,
rakenteeltaan ja ulkoasultaan.



\clearpage
%% L\"ahdeluettelo
%%
%% \phantomsection varmistaa, ett\"a hyperref-paketti latoo hypertekstilinkit
%% oikein.
%%
%% The \phantomsection command is nessesary for hyperref to jump to the
%% correct page, in other words it puts a hyper marker on the page.

\phantomsection
\addcontentsline{toc}{section}{\refname}
%\addcontentsline{toc}{section}{References}
\begin{thebibliography}{99}

%% Alla pilkun j\"alkeen on pakotettu oikea v\"ali \<v\"alily\"onti>-merkeill\"a.
\bibitem{Kauranen} Kauranen,\ I., Mustakallio,\ M. ja Palmgren,\ V.
  \textit{Tutkimusraportin kirjoittamisen opas opinn\"aytety\"on
    tekij\"oille.}  Espoo, Teknillinen korkeakoulu, 2006.

\bibitem{Itkonen} Itkonen,\ M. \textit{Typografian k\"asikirja.} 3.\
  painos.  Helsinki, RPS-yhti\"ot, 2007.

\bibitem{Koblitz} Koblitz,\ N. \textit{A Course in Number Theory and
    Cryptography. Graduate Texts in Mathematics 114.}  2.\ painos. New
  York, Springer, 1994.

%% Kun on useampi nimikirjain, jokaisen nimikirjaimen v\"aliin
%% kuuluu v\"alily\"onti. Oikea v\"alin m\"a\"ar\"a on saatu \<v\"alily\"onnill\"a>
\bibitem{bcs} Bardeen,\ J., Cooper,\ L.\ N. ja Schrieffer,\ J.\ R.
  Theory of Superconductivity. \textit{Physical Review,} 1957, vol.\
  108, nro~5, s.\ 1175--1204.

\bibitem{Deschamps} Deschamps,\ G.\ A. Electromagnetics and
  Differential Forms. \textit{Proceedings of the IEEE,} 1981, vol.\
  69, nro~6, s.\ 676--696.

%% Alla esimerkki englanninkielisen tavuttamisen pakottamisesta.
%% Oletusarvoisesti k\"aytet\"a\"an suomalaista tavutusta, mutta viitteiss\"a
%% esiintyy usein muunkielisi\"a lauseita, jotka tulevat siten tavutetuksi
%% suomen kielen s\"a\"ant\"ojen mukaan. T\"am\"an voi korjata \foreignlanguage-
%% komennolla, jonka ensimm\"ainen parametri on vieraan kielen nimi ja toinen
%% on vieraalla kielell\"a tavutettava teksti.
\bibitem{Sihvola} Sihvola,\ A.\ et al.
  Interpretation of measurements of helix
    and bihelix superchiral structures.
  Teoksessa: Jacob,\ A.\ F. ja
  Reinert,\ J. (toim.) \textit{Bianisotropics '98 7th International
    Conference on Complex Media.}  Braunschweig, 3.--6.6.1998.
  Braunscweig, Technische Universit\"at Braunschweig, 1998, s.\
  317--320.

%% Alla on suomalainen yhdistelm\"asukunimi. Sen nimien v\"aliss\"a
%% k\"aytet\"a\"an yhdysmerkki\"a l. tavuviivaa, kirjoitetaan -.
\bibitem{Lindblom} Lindblom-Yl\"anne,\ S. ja Wager,\ M.  Tieteellisten
  opinn\"aytet\"oiden ohjaaminen. Teoksessa: Lindblom-Yl\"anne,\ S. ja
  Nevgi,\ A. (toim.) \textit{Yliopisto- ja korkeakouluopettajan
    k\"asikirja.}  Helsinki, WSOY, 2004, s.\ 314--325.

\bibitem{Miinusmaa} Miinusmaa,\ H. Neliskulmaisen rei\"an poraamisesta
  kolmikulmaisella poralla. Diplomity\"o, Teknillinen korkeakoulu,
  konetekniikan osasto, Espoo, 1977.

%% T\"ass\"a taas pakotettu englanninkielinen tavutus.
%% Pedanttinen kirjoittaja pakottaa tietysti jokaiseen englanninkieliseen
%% lauseeseen englannin tavutuksen, mutta t\"ass\"a esityksess\"a ei n\"ain ole
%% tehty selvyyden ja l\"ahdekoodin luettavuuden takia.
\bibitem{Loh} Loh,\ N.\ C. High-Resolution Micromachined
  Interferometric Accelerometer. Master's Thesis, Massachusetts
  Institute of Technology, Cambridge,
  Massachusetts, 1992.

\bibitem{Lonnqvist} L\"onnqvist,\ A.
  Applications of hologram-based compact
    range: antenna radiation pattern, radar cross section, and
    absorber reflectivity measurements.
  V\"ait\"oskirja, Teknillinen korkeakoulu, s\"ahk\"o- ja tietoliikennetekniikan
  osasto, 2006.

\bibitem{sfs} SFS 5342. Kirjallisuusviitteiden laatiminen. 2.\ painos.
  Helsinki, Suomen standardisoimisliitto, 2004. 20~s.

\bibitem{haastattelu} Palmgren,\ V. Suunnittelija. Teknillinen
  korkeakoulu, kirjasto. Otaniementie 9, 02150 Espoo. Haastattelu
  15.1.2007.

\bibitem{Ribeiro} Ribeiro,\ C.\ B., Ollila,\ E. ja Koivunen,\ V.
  Stochastic Maximum-Likelihood Method for
    MIMO Propagation Parameter Estimation.
 \textit{IEEE Transactions
    on Signal Processing,} verkkolehti, vol.\ 55, nro~1, s.\ 46--55.
  Viitattu 19.1.2007. Lehti ilmestyy my\"os painettuna. DOI:
  10.1109/TSP.2006.882057.

\bibitem{Stieber} Stieber,\ T. GnuPG Hacks. \textit{Linux Journal,}
  verkkolehti, 2006, maaliskuu, nro~143. Viitattu 19.1.2007. Lehti
  ilmestyy my\"os painettuna. Saatavissa:
  \url{http://www.linuxjournal.com/article/8732.}

\bibitem{kone} Pohjois-Koivisto,\ T. Voiko kone tulevaisuudessa arvata
  tahtosi?  \textit{Apropos,} verkkolehti, helmikuu, nro~1, 2005.
  Viitattu 19.1.2007.  Saatavissa:
  \url{http://www.apropos.fi/1-2005/prima.php.}

\bibitem{Adida} Adida,\ B.  Advances in Cryptographic Voting Systems.
  Verkkodokumentti. Ph.D.\ Thesis, Massachusetts Institute of
  Technology, Cambridge,
  Massachusetts,
  2006. Viitattu 19.1.2007.  Saatavissa:
  \url{http://crypto.csail.mit.edu/~cis/theses/adida-phd.pdf.}

\bibitem{viittaaminen} Kilpel\"ainen,\ P. WWW-l\"ahteisiin viittaaminen
  tutkielmatekstiss\"a. Verkkodokumentti. P\"aivitetty 26.11.2001.
  Viitattu 19.1.2007. Saatavissa:
  \url{http://www.cs.uku.fi/~kilpelai/wwwlahteet.html.}

\end{thebibliography}

%% Appendices
%% Liitteet
\clearpage

\thesisappendix

\section{Esimerkki liitteest\"a\label{LiiteA}}

Liitteet eiv\"at ole opinn\"aytteen kannalta v\"altt\"am\"att\"omi\"a ja
opinn\"aytteen tekij\"an on
kirjoittamaan ryhtyess\"a\"an hyv\"a ajatella p\"arj\"a\"av\"ans\"a ilman liitteit\"a.
Kokemattomat kirjoittajat, jotka ovat huolissaan
tekstiosan pituudesta, paisuttavat turhan
helposti liitteit\"a pit\"a\"akseen tekstiosan pituuden annetuissa rajoissa.
T\"all\"a tavalla ei synny hyv\"a\"a opinn\"aytett\"a.

Liite on itsen\"ainen kokonaisuus, vaikka se t\"aydent\"a\"akin tekstiosaa.
Liite ei siten ole pelkk\"a listaus, kuva tai taulukko, vaan
liitteess\"a selitet\"a\"an aina sis\"all\"on laatu ja tarkoitus.

Liitteeseen voi laittaa esimerkiksi listauksia. Alla on
listausesimerkki t\"am\"an liitteen luomisesta.

%% Verbatim-ymp\"arist\"o ei muotoile tai tavuta teksti\"a. Fontti on monospace.
%% Verbatim-ymp\"arist\"on sis\"all\"a annettuja komentoja ei LaTeX k\"asittele.
%% Vasta \end{verbatim}-komennon j\"alkeen jatketaan k\"asittely\"a.
\begin{verbatim}
	\clearpage
	\appendix
	\addcontentsline{toc}{section}{Liite A}
	\section*{Liite A}
	...
	\thispagestyle{empty}
	...
	teksti\"a
	...
	\clearpage
\end{verbatim}

Kaavojen numerointi muodostaa liitteiss\"a oman kokonaisuutensa:
\begin{eqnarray}
d \wedge A  &=& F, \label{liitekaava1}\\
d \wedge F  &=& 0. \label{liitekaava2}
\end{eqnarray}


\clearpage
\section{Toinen esimerkki liitteest\"a\label{LiiteB}}

%% Liitteiden kaavat, taulukot ja kuvat numeroidaan omana kokonaisuutenaan
%%
%% Equations, tables and figures have their own numbering in Appendices
%\renewcommand{\theequation}{B\arabic{equation}}
%\setcounter{equation}{0}
%\renewcommand{\thefigure}{B\arabic{figure}}
%\setcounter{figure}{0}
%\renewcommand{\thetable}{B\arabic{table}}
%\setcounter{table}{0}

Liitteiss\"a voi my\"os olla kuvia, jotka
eiv\"at sovi leip\"atekstin joukkoon:
%% Ymp\"arist\"on figure parametrit htb pakottavat
%% kuvan t\"ah\"an, eik\"a LaTeX yrit\"a siirrell\"a niit\"a
%% hyv\"aksi katsomaansa paikkaan.
%% Ymp\"arist\"o\"a center voi k\"aytt\"a\"a \centering-
%% komennon sijaan
%%
%% Example of a figure, note the use of htb parameters which force
%% the figure to be inserted here
\begin{figure}[htb]
\begin{center}
\includegraphics[height=8cm]{kuva2}
\end{center}
\caption{Kuvateksti, jossa on liitteen numerointi}
\label{liitekuva}
\end{figure}
%%
Liitteiden taulukoiden numerointi on kuvien ja kaavojen kaltainen:
\begin{table}[htb]
\caption{Taulukon kuvateksti.}
\label{liitetaulukko}
\begin{center}
\fbox{
\begin{tabular}{lp{0.5\linewidth}}
9.00--9.55  & K\"aytett\"avyystestauksen tiedotustilaisuus (osanottajat
ovat saaneet s\"ahk\"opostitse valmistautumisteht\"av\"at, joten tiedotustilaisuus
voidaan pit\"a\"a lyhyen\"a).\\
9.55--10.00 & Testausalueelle siirtyminen
\end{tabular}}
\end{center}
\end{table}
Kaavojen numerointi muodostaa liitteiss\"a oman kokonaisuutensa:
\begin{eqnarray}
T_{ik} &=& -p g_{ik} + w u_i u_k + \tau_{ik},  \label{liitekaava3} \\
n_i    &=& n u_i + v_i.                      \label{liitekaava4}
\end{eqnarray}

\end{document}