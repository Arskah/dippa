%%%%%%%%%%%%%%%%%%%%%%%%%%%%%%%%%%%%%%%%%%%%%%%%%%%%%%%%%%%%%%%%%%%%
%%%%%%%%%%%%%%%%%%%%%%%%%%%%%%%%%%%%%%%%%%%%%%%%%%%%%%%%%%%%%%%%%%%%
%%                                                                %%
%% An example for writting your thesis using LaTeX                %%
%% Original version by Luis Costa,  changes by Perttu Puska       %%
%% Support for Swedish added 15092014                             %%
%%                                                                %%
%% This example consists of the files                             %%
%%         thesistemplate.tex (versio 2.01)                       %%
%%         opinnaytepohja.tex (versio 2.01) (for text in Finnish) %%
%%         aaltothesis.cls (versio 2.01)                          %%
%%         kuva1.eps                                              %%
%%         kuva2.eps                                              %%
%%         kuva1.pdf                                              %%
%%         kuva2.pdf                                              %%
%%                                                                %%
%%                                                                %%
%% Typeset either with                                            %%
%% latex:                                                         %%
%%             $ latex opinnaytepohja                             %%
%%             $ latex opinnaytepohja                             %%
%%                                                                %%
%%   Result is the file opinnayte.dvi, which                      %%
%%   is converted to ps format as follows:                        %%
%%                                                                %%
%%             $ dvips opinnaytepohja -o                          %%
%%                                                                %%
%%   and then to pdf as follows:                                  %%
%%                                                                %%
%%             $ ps2pdf opinnaytepohja.ps                         %%
%%                                                                %%
%% Or                                                             %%
%% pdflatex:                                                      %%
%%             $ pdflatex opinnaytepohja                          %%
%%             $ pdflatex opinnaytepohja                          %%
%%                                                                %%
%%   Result is the file opinnaytepohja.pdf                        %%
%%                                                                %%
%% Explanatory comments in this example begin with                %%
%% the characters %%, and changes that the user can make          %%
%% with the character %                                           %%
%%                                                                %%
%%%%%%%%%%%%%%%%%%%%%%%%%%%%%%%%%%%%%%%%%%%%%%%%%%%%%%%%%%%%%%%%%%%%
%%%%%%%%%%%%%%%%%%%%%%%%%%%%%%%%%%%%%%%%%%%%%%%%%%%%%%%%%%%%%%%%%%%%

%% Uncomment one of these:
%% the 1st when using pdflatex, which directly typesets your document in pdf (use jpg or pdf figures), or the 2nd when producing a ps file (use eps figures, don't use ps figures!).
\documentclass[english,12pt,a4paper,pdftex,sci,utf8]{aaltothesis}
%\documentclass[english,12pt,a4paper,dvips]{aaltothesis}

%% To the \documentclass above:
%% specify your school: arts, biz, chem, elec, eng, sci
%% specify the character encoding scheme used by your editor: utf8, latin1

%% Use one of these if you write in Finnish (see the Finnish template):
%\documentclass[finnish,12pt,a4paper,pdftex,elec,utf8]{aaltothesis}
%\documentclass[finnish,12pt,a4paper,dvips]{aaltothesis}

\usepackage{graphicx}

%% Use this if you write hard core mathematics, these are usually needed
\usepackage{amsfonts,amssymb,amsbsy,csquotes}

% \usepackage[backend=biber,natbib=true,style=numeric-comp,sorting=none]{biblatex}
\usepackage{biblatex}
\addbibresource{./refs.bib}

%% Use the macros in this package to change how the hyperref package below typesets its hypertext -- hyperlink colour, font, etc. See the package documentation. It also defines the \url macro, so use the package when not using the hyperref package.
%\usepackage{url}

%% Use this if you want to get links and nice output. Works well with pdflatex.
\usepackage{hyperref}
\hypersetup{pdfpagemode=UseNone, pdfstartview=FitH,
  colorlinks=true,urlcolor=red,linkcolor=blue,citecolor=black,
  pdftitle={Default Title, Modify},pdfauthor={Your Name},
  pdfkeywords={Modify keywords}}


%% All that is printed on paper starts here
\begin{document}

%% Change the school field to specify your school if the automatically set name is wrong
% \university{aalto-yliopisto}
% \university{aalto University}
% \school{Sähkötekniikan korkeakoulu}
% \school{School of Electrical Engineering}

%% Only for B.Sc. thesis: Choose your degree programme.
%\degreeprogram{Electronics and electrical engineering}

%% ONLY FOR M.Sc. AND LICENTIATE THESIS: Specify your department,
%% professorship and professorship code.
\department{Department of Computer Science}
\professorship{Computer Science}
%%

%% Valitse yksi näistä kolmesta
%%
%% Choose one of these:
%\univdegree{BSc}
\univdegree{MSc}
%\univdegree{Lic}

%% Your own name (should be self explanatory...)
\author{Aarni Halinen}

%% Your thesis title comes here and again before a possible abstract in
%% Finnish or Swedish . If the title is very long and latex does an
%% unsatisfactory job of breaking the lines, you will have to force a
%% linebreak with the \\ control character.
%% Do not hyphenate titles.
%%
\thesistitle{Kubernetes inter-pod container isolation}

\place{Espoo}

%% For B.Sc. thesis use the date when you present your thesis.
\date{1.6.2023}

%% B.Sc. or M.Sc. thesis supervisor
%% Note the "\" after the comma. This forces the following space to be a normal interword space, not the space that starts a new sentence.
%% This is done because the fullstop isn't the end of the sentence that should be followed by a slightly longer space but is to be followed by a regular space.
\supervisor{Prof.\ Mario Di Francesco} %{Prof.\ Pirjo Professori}

%% B.Sc. or M.Sc. thesis advisors(s). You can give upto two advisors in this template. Check with your supervisor how many official advisors you can have.
\advisor{M.Sc.\ (Tech.)\ José\ Luis\ Martin\ Navarro}
\advisor{M.Sc.\ (Tech.)\ Jacopo\ Bufalino}

%% Aalto logo: syntax:
%% \uselogo{aaltoRed|aaltoBlue|aaltoYellow|aaltoGray|aaltoGrayScale}{?|!|''}
%%
%% Logo language is set to be the same as the document language.
%% Logon kieli on sama kuin dokumentin kieli
\uselogo{aaltoRed}{''}

%% Create the coverpage
\makecoverpage

%% Note that when writting your master's thesis in English, place
%% the English abstract first followed by the possible Finnish abstract

%% English abstract.
%% All the information required in the abstract (your name, thesis title, etc.)
%% is used as specified above.
%% Specify keywords
\keywords{Kubernetes, Container, Docker, Security}
%% Abstract text
\begin{abstractpage}[english]
  Your abstract in English. Try to keep the abstract short; approximately
  100 words should be enough. The abstract explains your research topic,
  the methods you have used, and the results you obtained.
  Your abstract in English. Try to keep the abstract short; approximately
  100 words should be enough. The abstract explains your research topic,
  the methods you have used, and the results you obtained.

  Your abstract in English. Try to keep the abstract short; approximately
  100 words should be enough. The abstract explains your research topic,
  the methods you have used, and the results you obtained.
  Your abstract in English. Try to keep the abstract short; approximately
  100 words should be enough. The abstract explains your research topic,
  the methods you have used, and the results you obtained.
\end{abstractpage}

%% Force a new page so that the possible English abstract starts on a new page
\newpage
%
%% Abstract in Finnish.  Delete if you don't need it.
% \thesistitle{Opinnäyteohje}
% \advisor{TkT Olli Ohjaaja}
% \degreeprogram{Electronics and electrical engineering}
% \department{Radiotieteen ja -tekniikan laitos}
% \professorship{Piiriteoria}
% %% Avainsanat
% \keywords{Vastus, Resistanssi,\\ Lämpötila}
% %% Tiivistelmän tekstiosa
% \begin{abstractpage}[finnish]
%   Tiivistelmässä on lyhyt selvitys (noin 100 sanaa)
%   kirjoituksen tärkeimmästä sisällöstä: mitä ja miten on tutkittu,
%   sekä mitä tuloksia on saatu.
%   Tiivistelmässä on lyhyt selvitys (noin 100 sanaa)
%   kirjoituksen tärkeimmästä sisällöstä: mitä ja miten on tutkittu,
%   sekä mitä tuloksia on saatu.

%   Tiivistelmässä on lyhyt selvitys (noin 100 sanaa)
%   kirjoituksen tärkeimmästä sisällöstä: mitä ja miten on tutkittu,
%   sekä mitä tuloksia on saatu.
%   Tiivistelmässä on lyhyt selvitys (noin 100 sanaa)
%   kirjoituksen tärkeimmästä sisällöstä: mitä ja miten on tutkittu,
%   sekä mitä tuloksia on saatu.
%   Tiivistelmässä on lyhyt selvitys (noin 100 sanaa)
%   kirjoituksen tärkeimmästä sisällöstä: mitä ja miten on tutkittu,
%   sekä mitä tuloksia on saatu.
% \end{abstractpage}

% \newpage

\mysection{Preface}
I want to thank Professor Pirjo Professori
and my instructor Olli Ohjaaja for their
good and poor guidance.\\

\vspace{5cm}
Otaniemi, 16.1.2015

\vspace{5mm}
{\hfill Eddie E.\ A.\ Engineer \hspace{1cm}}

%% Force new page after preface
\newpage

%% Table of contents.
\thesistableofcontents

%% Symbols and abbreviations
\mysection{Symbols and abbreviations}

\subsection*{Symbols}

\begin{tabular}{ll}
$\uparrow$       & electron spin direction up\\
$\downarrow$     & electron spin direction down
\end{tabular}

\subsection*{Operators}

\begin{tabular}{ll}
$\nabla \times \mathbf{A}$              & curl of vectorin $\mathbf{A}$\\
\end{tabular}

\subsection*{Abbreviations}

\begin{tabular}{ll}
K8s         & Kubernetes \\
STRIDE      & an object-oriented analog circuit simulator and design tool \\
\end{tabular}


%% Tweaks the page numbering to meet the requirement of the thesis format:
%% Begin the pagenumbering in Arabian numerals (and leave the first page
%% of the text body empty, see \thispagestyle{empty} below).
%% Additionally, force the actual text to begin on a new page with the
%% \clearpage command.
%% \clearpage is similar to \newpage, but it also flushes the floats (figures
%% and tables).
%% There is no need to change these
\cleardoublepage
\storeinipagenumber
\pagenumbering{arabic}
\setcounter{page}{1}

\section{Introduction}

%% Leave first page empty
\thispagestyle{empty}

T\"am\"an tekstin l\"ahteen\"a oleva tiedosto on opinn\"aytteen
pohja, jota voi k\"aytt\"a\"a kandidaatinty\"oss\"a, diplomity\"oss\"a ja
lisensiaatinty\"oss\"a. Tekstin
l\"ahteen\"a oleva tiedosto on kirjoitettu  \LaTeX-tiedoston rakenteen
opiskelemista ajatellen. Tiedoston kommentit sis\"alt\"av\"at
tietoa, joka on hy\"odyllist\"a opinn\"aytett\"a kirjoitettaessa.

%% Esimerkki luettelosta. Lyhyt ajatusviiva on k\"ayt\"oss\"a
%% luettelossa, ja se on pituudeltaan
%% en dash. Merkit\"a\"an latex-koodissa --.
Johdanto selvitt\"a\"a samat asiat kuin tiivistelm\"a, mutta
laveammin. Johdannossa kerrotaan yleens\"a seuraavat asiat

\begin{itemize}
\item[--]Tutkimuksen taustaa ja tutkimusaiheen yleisluonteinen esittely
\item[--]Tutkimuksen tavoitteet
\item[--]P\"a\"akysymys ja osaongelmat
\item[--]Tutkimuksen rajaus ja keskeiset k\"asitteet.
\end{itemize}

Lyhyiden opinn\"aytteiden johdannot ovat yleens\"a liian pitki\"a, joten
johdannon paisuttamista on v\"altett\"av\"a. Diplomity\"oh\"on sopii johdanto,
joka on 2--4 sivua. %% t\"ass\"a on my\"os lyhyt ajatusviiva l. en dash.
Kandidaatinty\"on johdannon on oltava diplomity\"on
johdantoa lyhyempi. Sopivasti tiivistetty johdanto ei kaipaa alaotsikoita.


%% In a thesis, every section starts a new page, hence \clearpage
\clearpage

\section{Background}

\subsection{Principle of least priviledge, Zero trust...}

\subsection{Microservices?}
\subsection{Containerization and Docker}
\subsubsection{Docker components}
\subsubsection{Linux control groups and namespaces}
\subsubsection{Linux capabilities, priviledged containers, Container breakout}

\cite{bui2015analysis}

\begin{enumerate}
  \item Priviledged container
  \item CAP\_SYS\_ADMIN, mounting /proc and chroot
  \item CAP\_SYS\_PTRACE, shellcode injection to running program, nc 172.17.0.1 on port running shell
  \item Mounted docker socket, creating priviledged containers
\end{enumerate}


\subsection{Kubernetes system components, control plane}
\subsubsection{apiserver}
\subsubsection{etcd}
\subsubsection{scheduler}
\subsubsection{controller-manager}

\subsection{Kubernetes resources}
\subsubsection{Namespaces}
\subsubsection{Pods}
\subsubsection{Admission control}

\subsection{Kubernetes networking}
\subsubsection{Network policies}
\subsubsection{Container Network Interfaces}
\subsubsection{Cilium}
\subsubsection{eBPF}

\begin{enumerate}
  \item eXpress Data Path
  \item Traffic control
\end{enumerate}

\clearpage

\section{Research material and methods}

T\"ass\"a osassa kuvataan k\"aytetty tutkimusaineisto ja
tutkimuksen metodologiset valinnat, sek\"a
kerrotaan tutkimuksen toteutustapa ja k\"aytetyt menetelm\"at.

\clearpage

\section{Evaluation}

T\"ass\"a osassa esitet\"a\"an tulokset ja vastataan tutkielman alussa
esitettyihin tutkimuskysymyksiin. Tieteellisen kirjoitelman
arvo mitataan t\"ass\"a osassa esitettyjen tulosten perusteella.

Tutkimustuloksien merkityst\"a on aina syyt\"a arvioida ja tarkastella
kriittisesti.  Joskus tarkastelu voi olla t\"ass\"a osassa, mutta se
voidaan my\"os j\"att\"a\"a viimeiseen osaan, jolloin viimeisen osan nimeksi
tulee >>Tarkastelu>>. Tutkimustulosten merkityst\"a voi arvioida my\"os
>>Johtop\"a\"at\"okset>>-otsikon alla viimeisess\"a osassa.

T\"ass\"a osassa on syyt\"a my\"os arvioida tutkimustulosten luotettavuutta.
Jos tutkimustulosten merkityst\"a arvioidaan >>Tarkastelu>>-osassa,
voi luotettavuuden arviointi olla my\"os siell\"a.

\clearpage

\section{Conclusion}

Opinn\"aytteen tekij\"a vastaa siit\"a, ett\"a opinn\"ayte on t\"ass\"a dokumentissa
ja opinn\"aytteen tekemist\"a k\"asittelevill\"a luennoilla sek\"a
harjoituksissa annettujen ohjeiden mukainen muotoseikoiltaan,
rakenteeltaan ja ulkoasultaan.

\clearpage

%% The \phantomsection command is nessesary for hyperref to jump to the correct page, in other words it puts a hyper marker on the page.
\phantomsection

\thesisbibliography
\printbibliography

%% Appendices
\clearpage

\thesisappendix

\section{Esimerkki liitteest\"a\label{LiiteA}}

Liitteet eiv\"at ole opinn\"aytteen kannalta v\"altt\"am\"att\"omi\"a ja
opinn\"aytteen tekij\"an on
kirjoittamaan ryhtyess\"a\"an hyv\"a ajatella p\"arj\"a\"av\"ans\"a ilman liitteit\"a.
Kokemattomat kirjoittajat, jotka ovat huolissaan
tekstiosan pituudesta, paisuttavat turhan
helposti liitteit\"a pit\"a\"akseen tekstiosan pituuden annetuissa rajoissa.
T\"all\"a tavalla ei synny hyv\"a\"a opinn\"aytett\"a.

\clearpage

\end{document}