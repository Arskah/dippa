\documentclass{beamer}
\title{Securing Sidecar Containers in a Zero Trust Kubernetes Cluster}
\author{Aarni Halinen}
\institute{
  Department of Computer Science\\
  Aalto University
}
\date{\today}

\begin{document}

\frame{
\titlepage

Supervisor: Prof.\ Mario Di Francesco
\\
Advisor: M.Sc.\ (Tech.)\ José\ Luis\ Martin\ Navarro
\\
Advisor: M.Sc.\ (Tech.)\ Jacopo\ Bufalino
}

\begin{frame}
\frametitle{Zero Trust Architecture}

\begin{itemize}
  \item A
\end{itemize}
\end{frame}

\begin{frame}
\frametitle{Containers}

\begin{itemize}
  \item Linux namespaces
  \item Container breakout
\end{itemize}
\end{frame}

\begin{frame}
\frametitle{Kubernetes}

\begin{itemize}
  \item Industry standard container orchestrator
\end{itemize}
\end{frame}

\begin{frame}
\frametitle{Kubernetes Control Plane}

\begin{itemize}
  \item TODO: Add image
\end{itemize}
\end{frame}

\begin{frame}
\frametitle{Container Network Interface (CNI)}

\begin{itemize}
  \item Kubernetes has no built-in component that provides connectivity and reachability for Pods $\Rightarrow$ CNI plugins
  \item Plugins may differ in architecture significantly
  \item Calico and Cilium are powerful CNIs that work with IPTables and eBPF
\end{itemize}
\end{frame}

\begin{frame}
\frametitle{Network policies}

\begin{itemize}
  \item Default Kubernetes resouce, allows creationg of Ingress and Egress rules for Pods
  \item Rules are applied to Pod's default NIC by the CNI plugin
  \item Does not affect intra-Pod communication via loopback device
\end{itemize}
\end{frame}

\begin{frame}
\frametitle{Pod Security Admission control}

\begin{itemize}
  \item G
\end{itemize}
\end{frame}

\begin{frame}
\frametitle{Sidecar pattern}

\begin{itemize}
  \item H
\end{itemize}
\end{frame}

\begin{frame}
\frametitle{Sidecar pattern issues}

\begin{itemize}
  \item Containers within a Pod share namespace $\Rightarrow$ same PSA rules
  \item Same NetworkPolicy for all containers in a Pod $\Rightarrow$ all containers have same Egress policy
  \item No isolation on loopback device
\end{itemize}
\end{frame}

\begin{frame}
\frametitle{Network Isolation Solution 1: IPTables or eBPF}

\begin{itemize}
  \item Add IPTables rules to sidecar after Pod is deployed
  \item Containers share IP address $\Rightarrow$ need to use IPTables owner-module (user-id, similarly to Envoy proxy) to apply rules only for the sidecar
\end{itemize}
\end{frame}

\begin{frame}
  \frametitle{Network Isolation Solution 2: Split containers to own namespace}

  \begin{itemize}
    \item Own Pods $\Rightarrow$ NPs can be used to restrict communication
    \item Own Kubernetes namespaces $\Rightarrow$ own PSA rules
    \item Not sidecar anymore (no loopback connectivity, deployed to different Nodes...)
    \item Prevent scheduling on different Nodes with Node and Pod affinity
  \end{itemize}
  \end{frame}

\begin{frame}
\frametitle{Multus}

\begin{itemize}
  \item CNI plugin that allows multiple NICs per Pod
  \item Uses other CNI plugins for implementation the NICs
  \item Allows creation of another network segment between Pods
\end{itemize}
\end{frame}

\begin{frame}
\frametitle{Solution 2+: Re-routing loopback to Multus NIC}

\begin{itemize}
  \item Requires kernel flag to be set: net.ipv4.conf.all.route\_localnet=1
  \item DNAT rule on IPTables in sufficient but eBPF solution is also possible
  \item Open source Network Policy implementation for Multus network exists
  \item Combined with affinity, sidecar is deployed on same Node and communicates via loopback!
\end{itemize}
\end{frame}

\begin{frame}
\frametitle{Refining the solutions}

\begin{itemize}
  \item Use a custom CNI plugin or controller loop to modify Pod network namespaces, however no available implementations yet exist
  \item All solutions are somewhat hacky and unstable, including Multus network with policies
\end{itemize}
\end{frame}

\begin{frame}
\frametitle{Extra security}

\begin{itemize}
  \item Tetragon for observability?
  \item Service meshes?
\end{itemize}
\end{frame}

\begin{frame}
\frametitle{Re-cap}

\begin{itemize}
  \item Kubernetes does not support Zero Trust Architecture
  \item Custom CNIs and controllers can be used to build ZTA to Pod network namespace
\end{itemize}
\end{frame}

\begin{frame}
\frametitle{References}

\begin{itemize}
  \item N
\end{itemize}
\end{frame}

\end{document}
